\documentclass[11pt]{article}

% Basic packages
\usepackage{amsmath,amssymb,amsthm}
\usepackage{hyperref}
\usepackage{booktabs}
\usepackage{geometry}
\geometry{margin=1in}

% Title info
\title{Towards a Metric of Organization:\\
Exploratory Notes and Draft Framework}
\author{[You, me, collaborators]}
\date{August 2025}

\begin{document}
\maketitle

\begin{abstract}
Entropy is well formalized as a measure of uncertainty, but it does not capture
structure or organization. In these exploratory notes we sketch a complementary
metric---the Organization Index---that quantifies the efficiency of retrieval in
structured systems. We propose several candidate formulations, examine toy
examples, and study robustness under perturbations. The goal is to provide a
jumping-off point for a future theory of organization, parallel to information
theory.
\end{abstract}

\section{Motivation}
Entropy (Shannon entropy, or in physics, Boltzmann entropy) measures
unpredictability in a distribution. However, entropy is not synonymous with
structure: high entropy often corresponds to random noise, which is largely
useless without an external decoder. Human systems (data, libraries, puzzles,
computation) are valued for their organization: the speed and reliability with
which one can find what is needed.

\emph{Goal:} develop a complementary metric to entropy that quantifies
organization, defined operationally as the efficiency of retrieval.

\section{Conceptual Framework}
\begin{itemize}
  \item \textbf{Entropy:} Uncertainty or randomness in a source.
  \item \textbf{Organization (proposed):} Efficiency of retrieval.
\end{itemize}

\noindent
Key principle:
\[
\text{Organization} \;\equiv\; \frac{\text{Baseline cost}}{\mathbb{E}[T]}
\]
where $\mathbb{E}[T]$ is the expected number of steps to locate an element.

\section{Candidate Metrics}
\subsection{Organization Index vs. Sorted Baseline}
\[
\mathrm{OI}_{\log} = \frac{\log_2 n}{\mathbb{E}[T]}
\]
Here $n$ is the number of items. A perfectly sorted structure (binary search)
yields $\mathrm{OI}_{\log} \approx 1$. A linear scan yields
$\mathrm{OI}_{\log} \ll 1$. Hashing or indexing can yield $\mathrm{OI}_{\log} >
1$. Optionally, one may cap at 1.

\subsection{Entropy-Aware Organization}
\[
\mathrm{OI}_H = \frac{H(P)}{\mathbb{E}[T]}
\]
where $H(P)$ is Shannon entropy of the query distribution. Optimal search trees
achieve $\mathbb{E}[T] \approx H(P)+O(1)$. Thus $\mathrm{OI}_H \approx 1$ for
comparison-optimal organization, and $\mathrm{OI}_H > 1$ for auxiliary data
structures (e.g.\ hashing).

\subsection{Step-Discounted Organization (IR-inspired)}
Borrowing from information retrieval metrics such as NDCG:
\[
\mathrm{OI}_{\text{SDG}} = \mathbb{E}\!\left[\frac{1}{\log_2(1+T)}\right],
\]
which lies in $(0,1]$ and captures the diminishing returns of additional steps.

\subsection{Normalization of the Organization Index}
In many contexts it is desirable to bound the index to the unit interval $[0,1]$.
We define the normalized Organization Index so that:
\begin{itemize}
  \item Perfect disorder (linear scan, no exploitable structure) $\;\to\; 0$.
  \item Optimal comparison-based organization (binary search or entropy bound) $\;\to\; 1$.
\end{itemize}

Let $\mathbb{E}[T]_{\mathrm{rand}}$ denote the expected search cost under a
completely unorganized system (e.g.\ $(n+1)/2$ for linear scan under uniform
queries). Then:
\[
\mathrm{OI}^{\mathrm{norm}}_{\log} \;=\;
\frac{\tfrac{\log_2 n}{\mathbb{E}[T]} - \tfrac{\log_2 n}{\mathbb{E}[T]_{\mathrm{rand}}}}
{1 - \tfrac{\log_2 n}{\mathbb{E}[T]_{\mathrm{rand}}}}.
\]

Similarly, in the entropy-aware setting:
\[
\mathrm{OI}^{\mathrm{norm}}_H \;=\;
\frac{\tfrac{H(P)}{\mathbb{E}[T]} - \tfrac{H(P)}{\mathbb{E}[T]_{\mathrm{rand}}}}
{1 - \tfrac{H(P)}{\mathbb{E}[T]_{\mathrm{rand}}}}.
\]


\section{Toy Examples}
\subsection{Uniform queries, 52-card deck}
\begin{center}
\begin{tabular}{lcccc}
\toprule
Scenario & $\mathbb{E}[T]$ & $\mathrm{OI}_{\log}$ & $\mathrm{OI}^{\text{norm}}_{\log}$ & $\mathrm{OI}_{\text{SDG}}$ \\
\midrule
Unsorted pile   & $\sim26.5$ & 0.215 & 0.000 & 0.209 \\
Sorted (binary) & $\sim5.7$  & 1.000 & 1.000 & 0.364 \\
Hashed index    & $\sim1.2$  & 4.750 & 5.778 & 0.879 \\
\bottomrule
\end{tabular}
\end{center}

% Note: If you prefer a bounded score, use capped normalization
% OI^{norm,capped}_log = min(1, OI^{norm}_log), which maps the hashed row to 1.000.

\subsection{Skewed query distribution}
Suppose 50\% of queries target a single card. Then
$H(P)\approx 3.84$ bits and an optimal search tree achieves
$\mathbb{E}[T]\approx 3.84$, giving $\mathrm{OI}_H\approx 1$.

\section{Differential Organization}
\subsection{Misfiled Item}
If one item is placed in the wrong bucket of size $s$,
\[
\Delta \mathbb{E}[T] \;\approx\; \frac{1+\log_2(s+1)}{n}.
\]
Corresponding change in OI:
\[
\Delta \mathrm{OI} \;\approx\;
-\frac{\log_2 n}{(\log_2 n)^2}\cdot \frac{1+\log_2(s+1)}{n}.
\]

Example: 52-card deck sorted by suits. Perfect OI = 1. With one misfile,
$\mathrm{OI} \approx 0.984$. Smaller buckets are more robust.

\subsection{Reshaping Buckets}
Moving one item from bucket $a$ to $b$:
\[
\Delta \mathbb{E}[T] \;\approx\; \frac{1}{n\ln 2}
\left(\frac{1}{n_b} - \frac{1}{n_a}\right).
\]
Organization is maximized when buckets are balanced.

\subsection{Within-Bucket Disorder}
Binary search breaks under a single inversion; robust policies (guards +
fallback) reduce this to a misfile penalty.

\section{Properties of the Organization Index}
\paragraph{Not a metric.}
In the strict mathematical sense, $\mathrm{OI}$ is not a distance function and
does not obey the triangle inequality. It is better understood as an \emph{index}
(like entropy or the Gini coefficient). A derived dissimilarity
$|\mathrm{OI}(A)-\mathrm{OI}(B)|$ is symmetric but does not inherit metric
properties.

\paragraph{Monotonicity.}
$\mathrm{OI}$ is monotone in $\tfrac{1}{\mathbb{E}[T]}$. Fewer expected steps
always increases organization.

\paragraph{Scale dependence.}
Because the baseline grows as $\log n$, larger systems exhibit higher possible
organization scores. This matches intuition: a well-organized library is
``more organized'' than a well-organized deck of cards. Scale matters.

\paragraph{Boundedness.}
\begin{itemize}
  \item In comparison-only models, $\mathrm{OI}\leq 1$ (with small slack).
  \item With auxiliary data structures (hashing, indexing), $\mathrm{OI}$ can
  grow as $\Theta(\log n)$. Thus organization can scale unbounded with system
  size.
\end{itemize}

\paragraph{Differential sensitivity.}
Small perturbations (misfiles, bucket shifts) change $\mathrm{OI}$ smoothly,
with $\Delta \mathrm{OI}=O(1/n)$. Finer partitions are more robust to errors.

\section{Connections to Literature}
\begin{itemize}
  \item \textbf{Optimal search trees:} average cost bounded by entropy $H(P)$.
  \item \textbf{Huffman coding analogy:} minimizing expected code length mirrors
  minimizing expected search depth.
  \item \textbf{IR metrics:} NDCG and MRR parallel the step-discounting idea.
  \item \textbf{Complexity theory:} effective complexity, logical depth,
  statistical complexity. Our approach is operational and measurable.
\end{itemize}

\section{Open Questions and Future Work}
\begin{itemize}
  \item Normalization choice: bounded vs unbounded scores.
  \item Distributional sensitivity: handling skewed or unknown query
  distributions.
  \item Dynamic systems: including maintenance and update costs.
  \item Multi-level organization: hierarchies, caches, indexes.
  \item Biological analogy: SNP-like mutations as perturbations to structure.
  \item Applications: IR benchmarks, database indexing, cognitive models,
  evolutionary biology.
\end{itemize}

\section{Conclusion}
Entropy quantifies randomness; organization quantifies efficiency of access.
Our candidate definitions ($\mathrm{OI}_{\log}$, $\mathrm{OI}_H$,
$\mathrm{OI}_{\text{SDG}}$) provide complementary measures. Differential studies
reveal robustness properties, and scaling analysis separates comparison-based
organization from ``super-organized'' structures. This framework can grow into a
general \emph{theory of organization} parallel to information theory.

\bibliographystyle{plain}
% \bibliography{organization} % (future references)

\end{document}

